\section{Conclusions}

For the ROI, $b \in (\ang{-2},\ang{2}),\ \ell \in (\ang{-10},\ang{0})$, the total energy density in electrons with energy above $E_0 = \SI{1}{GeV}$ is given by the integral of the electron spectrum that was found in Section \ref{sec:IC_model}:

\be
\frac{\de E_\tot}{\de V} = \int_{E_0}^{\infty} \left(E \frac{\de N}{\de E}\right)_\el \de E = \SI{3.8e-14}{erg/cm^3}.
\ee
Assuming a distance of $\SI{8}{kpc}$ to the ROI, the volume of the ROI is $V = \SI{0.54}{kpc^3} = \SI{1.58e64}{cm^3}$. The total energy content of the ROI in electrons above $\SI{1}{GeV}$ is $E_\tot = \SI{6e50}{erg}$, which corresponds to the CR energy output of $60$ SNe.\\
Using the result from Section \ref{sec:Pion_model} we find an energy density in protons of $\de E_\tot / \de V = \SI{6.6e-13}{erg/cm^3}$ and a total energy content of $E_\tot = \SI{1e52}{erg}$. However, the gas density in the inner Galaxy is probably higher than $n_\Hy = \SI{1}{/cm^3}$ as assumed in Section \ref{sec:Pion_model}, resulting in an energy content in protons of the same order of magnitude as the energy density in electrons. \\
\\
To estimate the maximal propagation distance of electrons, we start with a diffusion equation taking into account diffusion and energy loss $b_\IC(E)$ via IC. From the solution we read off the diffusion distance for electrons with energy $E_0 = \SI{1}{TeV}$:
\be
\langle x \rangle^2 = 2 \int_{E_0}^E \frac{D(E)}{b_\IC(E)}\de E = \SI{1300}{pc}.
\ee
with a spatially constant diffusion coefficient $D(E) = D_0\left(\frac{E}{\SI{1}{GeV}}\right)^\delta$, where we take values of the local diffusion coefficient: $D_0 = \SI{3e28}{cm^2/s} = \SI{100}{pc^2/kyr}$ and $\delta = 0.4$. The energy loss $b_\IC(E)$ is \dots. Since the diffusion distance of the electrons exceeds the spatial size of the ROI, the electrons cannot be confined. Therefore, we find that a transient process is favoured. The energy losses of protons exceed the energy losses of electrons by far, therefore the same applies for protons.\\
Wit the local diffusion coefficient we find an escape time for both electrons and protons of 
\be
T = \frac{\Delta x^2}{2 D(E)} = \SI{70}{kyr}.
\ee

