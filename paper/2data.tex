\section{Data selection}
\lb{sec:data}


\begin{figure*}[h]
%\centering
\includegraphics[width=0.33\textwidth]{plots/Mollweide_data_source_range_0.pdf}
\includegraphics[width=0.33\textwidth]{plots/Mollweide_data_source_range_1.pdf}
\includegraphics[width=0.33\textwidth]{plots/Mollweide_data_source_range_2.pdf}
\caption{
SED intensity of the \Fermi-LAT 9 years of Source class data integrated in three energy ranges.
The maps are represented in logarithmic scale. 
The map in the energy range 10 -- 100 GeV (100 -- 1000 GeV) 
is smoothed with a Gaussian kernel with radius $\sigma = 0^\circ\!\!.5$ ($\sigma = 0^\circ\!\!.7$).
}
\label{fig:Maps_data}
\end{figure*}

%The main goal of the analysis is a study of a relatively small region $\lesssim 10^\circ$ from the GC for energies $\gtrsim 1$ GeV.
We use 9 years of the \Fermi-LAT Pass 8 Source class events
between August 4, 2008  and August 3, 2017 ({\Fermi} Mission Elapsed Time 239557418\,s -- 523411376\,s)
with energies between 316 MeV $ = 10^{2.5}$ MeV
and 1 TeV separated in 21 logarithmic energy bins (6 bins per decade).
The selection of the events is performed with the standard quality cuts (DATA\_QUAL$>$0)\&\&(LAT\_CONFIG==1).
In order to avoid contamination from gamma rays produced in interactions of cosmic ray in the Earth atmosphere, 
we select events with the zenith angles $\theta < 100^{\circ}$,
which is sufficient for energies above 316 MeV.
We calculate the exposure and point-spread function (PSF) using the {\Fermi}-LAT Science Tools package version 
10-01-01 available from the {\Fermi} Science Support Center\footnote{\url{http://fermi.gsfc.nasa.gov/ssc/data/analysis/}} 
with the P8R2\_SOURCE\_V6 instrument response functions.
Figure \ref{fig:Maps_data} shows the data in units of integrated flux in three energy ranges. For all maps shown in this paper, we use Galactic coordinates centered on the GC in Mollweide projection. %The graticule spacing is $\ang{10}$ in latitude and longitude. 
%The maps are smoothed with a $\sigma = \ang{0.5}$ Gaussian kernel.
For spatial binning we use HEALPix\footnote{\url{http://sourceforge.net/projects/healpix/}} \citep{2005ApJ...622..759G} scheme with a pixelization of order 7  ($\approx 0\degr\!\!.46$ pixel size). 

