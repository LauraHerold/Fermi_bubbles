\section{Introduction}
\lb{sec:intro}



The \Fermi bubbles (FBs) are one of the most spectacular and unexpected discoveries 
in the \Fermi Large Area Telescope (LAT) data \citep{2010ApJ...724.1044S}.
The FBs extend to $55^\circ$ above and below the Galactic center (GC),
they have a well-defined edge and a relatively uniform intensity across the surface, apart from a ``cocoon'' in the south eastern part of the bubbles
\citep{2012ApJ...753...61S, 2014ApJ...793...64A}.
The intensity spectrum is $\sim E^{-2}$ at GeV energies with a cutoff or a softening around 100 GeV at latitudes $|b| > 10^\circ$ \citep{2014ApJ...793...64A}.
The origin of the FBs is attributed either to an emission from the supermassive black hole (SMBH) at
the center of our Galaxy %(AGN scenario)
or to a period of starburst activity which resulted in a combined wind
from supernova (SN) explosions of massive stars %(starburst scenario),  
\citep{2010ApJ...724.1044S}.
The gamma-ray signal up to 100 GeV can be produced either by interactions of hadronic cosmic rays (CR) with gas (hadronic model)
or by inverse Compton (IC) scattering of high energy electrons and positrons and the interstellar radiation (leptonic model).

Although the FBs were detected about 8 years ago, their origin is still unresolved.
Important insights into their origin can be obtained from the study of the morphology and the spectrum of the FBs near the GC.
%Understanding the origin of the FBs will provide a unique opportunity to test the predictions of numerical simulations  of either the emission from the SMBH or the starburst activity near the GC.
%A study of the gamma-ray emission near the GC is important to understand the origin of the FBs.
The spectrum of gamma rays can provide information on the composition of the 
CR that produce the gamma-ray signal (hadronic vs leptonic CR),
as well as the age of the CR (through a cooling cutoff in the leptonic scenario or a break due to escape of high energy CR)
and the spectrum of the CR at the source.
The morphology of the emission can point to the source of the bubbles: either the SMBH Sgr A* or a recent star-forming region.
Previous analyses of the FBs at low latitudes indicated higher intensity of emission near the Galactic plane (GP) and a displacement
to negative longitudes \citep{2016ApJS..223...26A, 2017ApJ...840...43A, 2017JCAP...08..022S}.
The spectrum of the FBs for $|b| < 10^\circ$ is consistent with a power-law $\propto E^{-2}$ 
without a cutoff up to 1 TeV \citep{2017ApJ...840...43A}.
%If confirmed, the high intensity and the hard spectrum of the FBs at low latitudes will open up a possibility of a detection of the FBs with current and future imaging atmospheric Cherenkov telescopes (IACTs) and with neutrino telescopes, which will further constrain models of the FBs formation.

The study of the FBs in the GP is complicated due to bright Galactic diffuse emission components.
The $\pi^0$ component of the gamma-ray emission is well traced by the distribution of gas,
but it has large uncertainties towards the GC due to a lack of kinematic information from the motion of the gas in the 
Galaxy, which is used to reconstruct the gas distribution
(the velocity of the gas in the direction of the GC is perpendicular to the line of sight)
as well as the uncertainties in the CO emission along the line of sight to the GC (which is used as a tracer of molecular hydrogen)
and large dispersion of velocities of some molecular clouds near the GC. 
There are also large uncertainties in the distribution of the CR sources and the propagation model near the GC,
which make it rather difficult to predict a priori 
%, even using the standard tracers, such as supernova remnants (SNRs), 
the distribution of the propagated CR in the Galaxy.
For the IC component of the gamma-ray emission, 
there is a significant uncertainty in the interstellar radiation field (ISRF) density near the GC \citep[e.g.,][]{2017ApJ...846...67P} in addition to 
the uncertainties in the CR distribution.
There should also be undetected point-like and extended sources, which nevertheless contribute to the total flux.
Distribution of CR in the Galaxy can be computed with CR propagation tools, such as GALPROP \citep{2007ARNPS..57..285S}.
The maps of gamma-ray emission from interactions of CR with gas in different Galactocentric rings and from interaction
of CR electrons and positrons with the ISRF can be used as templates for the corresponding components
of gamma-ray emission.
The agreement of the gamma-ray data with models based on templates derived with the CR propagation tools
is rather poor in the GP %from the statistical point of view 
\citep[e.g.,][]{2012ApJ...750....3A, 2017ApJ...840...43A}.

In this paper, we analyze the gamma-ray emission at the base of the FBs and 
estimate the uncertainties %in the residual hard and bright component
%is an analysis of the FBs at low latitudes to estimate the uncertainties on the gamma-ray emission 
related to modeling of the Galactic foreground and background components.
We focus on morphology and spectrum of the FBs at energies from 10 GeV to 1 TeV,
where the intensity of the gamma-ray emission from the FBs 
relative to the other Galactic components
is higher than at low energies due to softer spectra of the Galactic components.
The study of the FBs at high energies will be also important for future searches with neutrino and Cherenkov telescopes.

%which we subtract from the data to find the gamma-ray emission from the FBs.
%Since the Galactic gamma-ray emission has large uncertainties towards the GC \citep[e.g.,][]{Calore:2014xka, 2017ApJ...840...43A},
We use several methods to determine the Galactic foreground/background emission.
Unfortunately the notion of the foreground and background emission may not be well defined if we search for
an extended gamma-ray emitting source based on the gamma-ray data itself
(rather than using the multi-wavelength data, e.g., to trace the distribution of gas).
Here and in the following by foreground/background emission we will mean a steady state 
(or average) diffuse emission of gamma-rays.
The steady state distribution of CR is obtained by averaging in time over many sources, it results in the local CR proton spectrum
of $\sim E^{-2.7}$ for energies from a few GeV to about a PeV and $\sim E^{-3.3}$ for the local CR electrons from GeV to TeV energies.
If the spectrum of the CR at the source is $\sim E^{-2.0 - 2.2}$, then the softening of the spectrum of CR protons by $\sim E^{-0.3} - E^{-0.6}$ 
is due to energy-dependent escape from the Galaxy,
while the softening for the CR electrons by $\sim E^{-1}$ is due to cooling.
A distinguishing property of a source of CR is that the spectrum of CR at or near the source is much harder than the average (propagated)
spectrum of CR.
Thus, one can look for the presence of a population of freshly accelerated CR by searching for areas of gamma-ray emission
with spectrum harder than average. In particular, one can use an ``on-off'' analysis to subtract the stationary component of the gamma-ray emission.
One of the caveats of this analysis is that the stationary gamma-ray emission in the ``on'' region can be more intense than the emission in the ``off'' region,
but in this case the difference of the fluxes would have a soft spectrum characteristic of the propagated CR,
unless the difference in flux is much smaller than the flux in both ``on'' and  ``off''  regions: in this case the difference can have a harder spectrum than the two terms.
We use the ``on-off'' technique as a preliminary check in a search for a population of freshly accelerated CR:
the hard spectrum of the difference is a necessary condition for the presence of a population of CR with a spectrum harder than the stationary distribution of CR.
It is also a sufficient condition for the presence of a population of CR with a hard spectrum, if the difference
has intensity comparable to the overall intensity in the ``on'' and  ``off''  regions.

%The goal is to test the uncertainty in the foreground and background emission in the determination of  the gamma-ray flux at the base of the FBs.
Consequently, on consideration that there is a tentative asymmetry in diffuse gamma-ray emission near the GC at high energies with a spectrum
harder to the west of the GC relative to the emission to the east of the GC,
as a first step,
we estimate the amount of the asymmetry by taking the difference in the gamma-ray data to the west and to the east of the GC 
after masking bright point sources.
Second, since the spectrum of the FBs at high latitudes as well as the inferred spectrum at low latitudes
is harder than the spectra of the other components of diffuse emission,
the contribution of the FBs at energies $\lesssim 1$ GeV is relatively small. 
We use the data below 1 GeV as a template of the Galactic foreground emission, 
which we fit to the data at energies above 1 GeV.
The residual emission is used to estimate the contribution from the components with spectra harder than the typical 
spectra of the stationary diffuse gamma-ray components.
Finally, we determine a model for the Galactic gamma-ray diffuse emission using templates for the emission 
components based on GALPROP calculation%
\footnote{\url{http://galprop.stanford.edu}} 
\citep{Moskalenko:1997gh, Strong:1998fr, Strong:2004de, Ptuskin:2005ax, 2007ARNPS..57..285S, Porter:2008ve,Vladimirov:2010aq}. 
In this model, we allow many free parameters, such as rescaling of the $\pi^0$ and bremsstrahlung emission in Galactocentric rings and refitting of bright point sources near the GC.
With the many free parameters, this model absorbs as much of the gamma-ray emission at the base of the FBs as it can. As a result, the residual emission at the base of the FBs in this model is smaller than the residuals in the other models considered in the paper.

The similarity of the energy spectrum below 100 GeV of the hard and bright emission at the base of the FBs 
and the FBs at high latitudes as well as the spatial coincidence of the emission with a continuation of the FBs from high latitudes 
make the physical correspondence of the two objects very plausible.
Nevertheless, their alignment along the line of sight can be accidental and the distance to the two objects can be different, i.e., the FBs may be
above and below the GC while the hard component may be at a closer distance, e.g., 1 kpc, from us so that the two objects would be physically unrelated.
Thus, we will refer to the hard and bright component as ``emission at the base of the FBs'' to keep both possibilities open: 
that the emission is the base of the FBs and that the emission is at a different location along the line of sight towards the base of the FBs.
We discuss possible origins of the hard component of emission in Section \ref{sec:Interpretation}.
Section \ref{sect:concl} contains conclusions.
