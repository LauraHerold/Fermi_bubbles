\documentclass[preprint]{aa}
\pdfoutput=1

\bibliographystyle{aa}
%\AuthorCallLimit=200


\usepackage{amsmath}
\usepackage{amsfonts}
\usepackage{dsfont}
\usepackage{amsxtra}
\usepackage{hyperref}
\usepackage{amssymb}
\usepackage{upgreek}

\usepackage{multirow}
\usepackage{url}

\usepackage{graphicx,epsfig}
%\usepackage{dcolumn}
%\usepackage{bm}
%\usepackage{ulem}

\usepackage{xspace} 

\usepackage{color}
\usepackage{xcolor}

\usepackage{lineno}
%\linenumbers

\usepackage{subcaption}
\usepackage{graphicx} % for subfigures

\renewcommand{\baselinestretch}{1.2}


%\setlength{\textwidth}{17.5cm} 
%\setlength{\textheight}{24.cm}
%\setlength{\topmargin}{-1.5cm}
%\setlength{\oddsidemargin}{-0.4cm}


\newcommand{\be}{\begin{equation}}
\newcommand{\ee}{\end{equation}}
\newcommand{\bea}{\begin{eqnarray}}
\newcommand{\eea}{\end{eqnarray}}
\newcommand{\beaa}{\begin{eqnarray*}}
\newcommand{\eeaa}{\end{eqnarray*}}
\newcommand{\ba}{\begin{array}}
\newcommand{\ea}{\end{array}}
\newcommand{\bi}{\begin{itemize}}
\newcommand{\ei}{\end{itemize}}
\newcommand{\ben}{\begin{enumerate}}
\newcommand{\een}{\end{enumerate}}

\newcommand{\bra}{\langle}
\newcommand{\ket}{\rangle}
\newcommand{\ra}{\rightarrow}
\newcommand{\lra}{\longrightarrow}
\newcommand{\overar}{\overrightarrow}
\newcommand{\wt}{\widetilde}
\newcommand{\td}{\tilde}


\newcommand{\lb}{\label}
\newcommand{\g}{\ensuremath{\gamma}\xspace}
\newcommand{\G}{\Gamma}
\newcommand{\e}{\epsilon}
\newcommand{\al}{\alpha}
\newcommand{\bt}{\beta}
\newcommand{\p}{\partial}
\newcommand{\dl}{\delta}
\newcommand{\Dl}{\Delta}
\newcommand{\ld}{\lambda}
\newcommand{\Ld}{\Lambda}
\newcommand{\vp}{\varphi}
\newcommand{\te}{\theta}
\newcommand{\Om}{\Omega}
\newcommand{\om}{\omega}
\newcommand{\sm}{\sigma}
\newcommand{\Sm}{\Sigma}

\newcommand{\A}{{\rm A}}
\newcommand{\B}{{\rm B}}
\newcommand{\U}{{\rm U}}
\newcommand{\F}{{\rm F}}
\newcommand{\SU}{{\rm SU}}
\newcommand{\Tr}{{\rm Tr}}
\newcommand{\Hom}{{\rm Hom}}

\newcommand{\FF}{{\mathsf F}}

\newcommand{\mcE}{{\mathcal{E}}}
\newcommand{\N}{{\mathcal{N}}}
\newcommand{\D}{{\mathcal{D}}}
\newcommand{\La}{{\mathcal{L}}}
\newcommand{\OO}{{\mathcal{O}}}
\newcommand{\M}{{\mathcal{M}}}


\newcommand{\mbE}{{\mathbb{E}}}
\newcommand{\Z}{{\mathbb{Z}}}
\newcommand{\R}{{\mathbb{R}}}
\newcommand{\C}{{\mathbb{C}}}
\newcommand{\NN}{{\mathbb{N}}}
\newcommand{\PP}{{\mathbb{C}}{\rm P}}

\newcommand{\HH}{{\mathcal{H}}}
\newcommand{\Hd}{{\mathcal{H}}^*}

\newcommand{\HI}{H~\textsc{i}\xspace}
\newcommand{\Htwo}{$\mathrm{H}_2$\xspace}
\newcommand{\hi}{$\mathrm{H\,\scriptstyle{I}}$\xspace}
\newcommand{\hii}{$\mathrm{H\,\scriptstyle{II}}$\xspace}
\newcommand{\hd}{$\mathrm{H}_2$\xspace}
\newcommand{\xco}{$X_\mathrm{CO}$\xspace}

\newcommand{\vx}{{\bf x}}

\newcommand{\Fermi}{\textit{Fermi}\xspace}
\newcommand{\LAT}{\textsl{LAT}\xspace}
\newcommand{\WMAP}{\textsl{WMAP}\xspace}
\newcommand{\Planck}{\textsl{Planck}\xspace}
\newcommand{\Suzaku}{\textsl{Suzaku}\xspace}

\newcommand{\SM}{Sample Model\xspace}

\newcommand{\sigmav}{\ensuremath{\langle \sigma v \rangle}\xspace}
\newcommand{\bbbar}{\ensuremath{b \bar b}\xspace}
\newcommand{\tautau}{\ensuremath{\tau^{+}\tau^{-}}\xspace}
\newcommand{\relic}{\ensuremath{2.2\times10^{-26}\cm^{3}\second^{-1}}\xspace}
\newcommand{\beff}{\ensuremath{b_{\rm eff}}\xspace}
\newcommand{\DM}{\ensuremath{\mathrm{DM}}}
\newcommand{\mDM}{\ensuremath{m_\DM}\xspace}


% local options
\newcommand{\onepic}{0.45}
\newcommand{\twopic}{0.38}
\newcommand{\twopicsp}{0.45}
\newcommand{\threepic}{0.25}
%\newcommand{\threepic}{0.18}
\newcommand{\fourpic}{0.28}
\newcommand{\twopicwca}{0.35}

\newcommand{\cmap}{_afmhot}


\newcommand{\red}{\textcolor{red}}
\newcommand{\blue}{\textcolor{blue}}
\definecolor{darkgreen}{rgb}{0.0, 0.7, 0.0}
\newcommand{\green}{\textcolor{darkgreen}}

\newcommand{\dima}[1]{\textcolor{blue}{(Dima: #1)}}
\newcommand{\Laura}[1]{\textcolor{red}{(Laura: #1)}}
\newcommand{\low}{\text{low}}
\newcommand{\model}{\text{model}}
% local options

\begin{document} 


   \title{A study of the \Fermi bubbles near the Galactic plane}

   %\subtitle{}

   \author{L. Herold \thanks{\email{laura.herold@fau.de}}
          \inst{1}
          \and
          D. Malyshev \thanks{\email{dmitry.mayshev@fau.de}}
          \inst{1}
          }

   \institute{
             Erlangen Centre for Astroparticle Physics, Erwin-Rommel-Str. 1, Erlangen, Germany
             }

   \date{Received September 15, 1996; accepted March 16, 1997}

% \abstract{}{}{}{}{} 
% 5 {} token are mandatory
 
  \abstract
  % context heading (optional)
  % {} leave it empty if necessary  
   {\Fermi bubbles are one of the most unexpected discoveries by the \Fermi LAT. 
   At the moment the origin of the bubbles and the nature of the gamma-ray emission are still unresolved questions.
   The behavior of the bubbles near the Galactic plane may have an important clue to solve the question of the origin of the bubbles.}
  % aims heading (mandatory)
   {We analyze 8 years of the \Fermi LAT pass 8 data to study the morphology and spectrum of the \Fermi bubbles near the Galactic plane.
   }
  % methods heading (mandatory)
   {We use several methods to separate the emission from the bubbles from the Galactic foreground diffuse emission and the contribution from
   point sources.}
  % results heading (mandatory)
   {We confirm that the \Fermi bubbles have a higher intensity of gamma-ray emission near the GP relative to high latitudes,
   the emission is shifted to the west (negative longitudes) from the GC. 
   The spectrum is consistent with a single power law up to 1 TeV}
  % conclusions heading (optional), leave it empty if necessary 
   {}

   \keywords{Gamma rays: general --
                Galaxy: center --
                Galaxy: halo --
                %Galaxy: structure -- 
                ISM: jets and outflows
               }

\maketitle
   
   
\tableofcontents


\section{Introduction}
\lb{sec:intro}



The \Fermi bubbles (FBs) are one of the most spectacular and unexpected discoveries 
in the \Fermi Large Area Telescope (LAT) data \citep{2010ApJ...724.1044S}.
The FBs extend to $55^\circ$ above and below the Galactic center (GC),
they have a well-defined edge and a relatively uniform intensity across the surface, apart from a ``cocoon'' in the south eastern part of the bubbles
\citep{2012ApJ...753...61S, 2014ApJ...793...64A}.
The intensity spectrum is $\sim E^{-2}$ at GeV energies with a cutoff or a softening around 100 GeV at latitudes $|b| > 10^\circ$ \citep{2014ApJ...793...64A}.
The origin of the FBs is attributed either to an emission from the supermassive black hole (SMBH) at
the center of our Galaxy %(AGN scenario)
or to a period of starburst activity which resulted in a combined wind
from supernova (SN) explosions of massive stars %(starburst scenario),  
\citep{2010ApJ...724.1044S}.
The gamma-ray signal up to 100 GeV can be produced either by interactions of hadronic cosmic rays (CR) with gas (hadronic model)
or by inverse Compton (IC) scattering of high energy electrons and positrons and the interstellar radiation (leptonic model).

Although the FBs were detected about 8 years ago, their origin is still unresolved.
Important insights into their origin can be obtained from the study of the morphology and the spectrum of the FBs near the GC.
%Understanding the origin of the FBs will provide a unique opportunity to test the predictions of numerical simulations  of either the emission from the SMBH or the starburst activity near the GC.
%A study of the gamma-ray emission near the GC is important to understand the origin of the FBs.
The spectrum of gamma rays can provide information on the composition of the 
CR that produce the gamma-ray signal (hadronic vs leptonic CR),
as well as the age of the CR (through a cooling cutoff in the leptonic scenario or a break due to escape of high energy CR)
and the spectrum of the CR at the source.
The morphology of the emission can point to the source of the bubbles: either the SMBH Sgr A* or a recent star-forming region.
Previous analyses of the FBs at low latitudes indicated higher intensity of emission near the Galactic plane (GP) and a displacement
to negative longitudes \citep{2016ApJS..223...26A, 2017ApJ...840...43A, 2017JCAP...08..022S}.
The spectrum of the FBs for $|b| < 10^\circ$ is consistent with a power-law $\propto E^{-2}$ 
without a cutoff up to 1 TeV \citep{2017ApJ...840...43A}.
%If confirmed, the high intensity and the hard spectrum of the FBs at low latitudes will open up a possibility of a detection of the FBs with current and future imaging atmospheric Cherenkov telescopes (IACTs) and with neutrino telescopes, which will further constrain models of the FBs formation.

The study of the FBs in the GP is complicated due to bright Galactic diffuse emission components.
The $\pi^0$ component of the gamma-ray emission is well traced by the distribution of gas,
but it has large uncertainties towards the GC due to a lack of kinematic information from the motion of the gas in the 
Galaxy, which is used to reconstruct the gas distribution
(the velocity of the gas in the direction of the GC is perpendicular to the line of sight)
as well as the uncertainties in the CO emission along the line of sight to the GC (which is used as a tracer of molecular hydrogen)
and large dispersion of velocities of some molecular clouds near the GC. 
There are also large uncertainties in the distribution of the CR sources and the propagation model near the GC,
which make it rather difficult to predict a priori 
%, even using the standard tracers, such as supernova remnants (SNRs), 
the distribution of the propagated CR in the Galaxy.
For the IC component of the gamma-ray emission, 
there is a significant uncertainty in the interstellar radiation field (ISRF) density near the GC \citep[e.g.,][]{2017ApJ...846...67P} in addition to 
the uncertainties in the CR distribution.
There should also be undetected point-like and extended sources, which nevertheless contribute to the total flux.
Distribution of CR in the Galaxy can be computed with CR propagation tools, such as GALPROP \citep{2007ARNPS..57..285S}.
The maps of gamma-ray emission from interactions of CR with gas in different Galactocentric rings and from interaction
of CR electrons and positrons with the ISRF can be used as templates for the corresponding components
of gamma-ray emission.
The agreement of the gamma-ray data with models based on templates derived with the CR propagation tools
is rather poor in the GP %from the statistical point of view 
\citep[e.g.,][]{2012ApJ...750....3A, 2017ApJ...840...43A}.

In this paper, we analyze the gamma-ray emission at the base of the FBs and 
estimate the uncertainties %in the residual hard and bright component
%is an analysis of the FBs at low latitudes to estimate the uncertainties on the gamma-ray emission 
related to modeling of the Galactic foreground and background components.
We focus on morphology and spectrum of the FBs at energies from 10 GeV to 1 TeV,
where the intensity of the gamma-ray emission from the FBs 
relative to the other Galactic components
is higher than at low energies due to softer spectra of the Galactic components.
The study of the FBs at high energies will be also important for future searches with neutrino and Cherenkov telescopes.

%which we subtract from the data to find the gamma-ray emission from the FBs.
%Since the Galactic gamma-ray emission has large uncertainties towards the GC \citep[e.g.,][]{Calore:2014xka, 2017ApJ...840...43A},
We use several methods to determine the Galactic foreground/background emission.
Unfortunately the notion of the foreground and background emission may not be well defined if we search for
an extended gamma-ray emitting source based on the gamma-ray data itself
(rather than using the multi-wavelength data, e.g., to trace the distribution of gas).
Here and in the following by foreground/background emission we will mean a steady state 
(or average) diffuse emission of gamma-rays.
The steady state distribution of CR is obtained by averaging in time over many sources, it results in the local CR proton spectrum
of $\sim E^{-2.7}$ for energies from a few GeV to about a PeV and $\sim E^{-3.3}$ for the local CR electrons from GeV to TeV energies.
If the spectrum of the CR at the source is $\sim E^{-2.0 - 2.2}$, then the softening of the spectrum of CR protons by $\sim E^{-0.3} - E^{-0.6}$ 
is due to energy-dependent escape from the Galaxy,
while the softening for the CR electrons by $\sim E^{-1}$ is due to cooling.
A distinguishing property of a source of CR is that the spectrum of CR at or near the source is much harder than the average (propagated)
spectrum of CR.
Thus, one can look for the presence of a population of freshly accelerated CR by searching for areas of gamma-ray emission
with spectrum harder than average. In particular, one can use an ``on-off'' analysis to subtract the stationary component of the gamma-ray emission.
One of the caveats of this analysis is that the stationary gamma-ray emission in the ``on'' region can be more intense than the emission in the ``off'' region,
but in this case the difference of the fluxes would have a soft spectrum characteristic of the propagated CR,
unless the difference in flux is much smaller than the flux in both ``on'' and  ``off''  regions: in this case the difference can have a harder spectrum than the two terms.
We use the ``on-off'' technique as a preliminary check in a search for a population of freshly accelerated CR:
the hard spectrum of the difference is a necessary condition for the presence of a population of CR with a spectrum harder than the stationary distribution of CR.
It is also a sufficient condition for the presence of a population of CR with a hard spectrum, if the difference
has intensity comparable to the overall intensity in the ``on'' and  ``off''  regions.

%The goal is to test the uncertainty in the foreground and background emission in the determination of  the gamma-ray flux at the base of the FBs.
Consequently, on consideration that there is a tentative asymmetry in diffuse gamma-ray emission near the GC at high energies with a spectrum
harder to the west of the GC relative to the emission to the east of the GC,
as a first step,
we estimate the amount of the asymmetry by taking the difference in the gamma-ray data to the west and to the east of the GC 
after masking bright point sources.
Second, since the spectrum of the FBs at high latitudes as well as the inferred spectrum at low latitudes
is harder than the spectra of the other components of diffuse emission,
the contribution of the FBs at energies $\lesssim 1$ GeV is relatively small. 
We use the data below 1 GeV as a template of the Galactic foreground emission, 
which we fit to the data at energies above 1 GeV.
The residual emission is used to estimate the contribution from the components with spectra harder than the typical 
spectra of the stationary diffuse gamma-ray components.
Finally, we determine a model for the Galactic gamma-ray diffuse emission using templates for the emission 
components based on GALPROP calculation%
\footnote{\url{http://galprop.stanford.edu}} 
\citep{Moskalenko:1997gh, Strong:1998fr, Strong:2004de, Ptuskin:2005ax, 2007ARNPS..57..285S, Porter:2008ve,Vladimirov:2010aq}. 
In this model, we allow many free parameters, such as rescaling of the $\pi^0$ and bremsstrahlung emission in Galactocentric rings and refitting of bright point sources near the GC.
With the many free parameters, this model absorbs as much of the gamma-ray emission at the base of the FBs as it can. As a result, the residual emission at the base of the FBs in this model is smaller than the residuals in the other models considered in the paper.

The similarity of the energy spectrum below 100 GeV of the hard and bright emission at the base of the FBs 
and the FBs at high latitudes as well as the spatial coincidence of the emission with a continuation of the FBs from high latitudes 
make the physical correspondence of the two objects very plausible.
Nevertheless, their alignment along the line of sight can be accidental and the distance to the two objects can be different, i.e., the FBs may be
above and below the GC while the hard component may be at a closer distance, e.g., 1 kpc, from us so that the two objects would be physically unrelated.
Thus, we will refer to the hard and bright component as ``emission at the base of the FBs'' to keep both possibilities open: 
that the emission is the base of the FBs and that the emission is at a different location along the line of sight towards the base of the FBs.
We discuss possible origins of the hard component of emission in Section \ref{sec:Interpretation}.
Section \ref{sect:concl} contains conclusions.

\section{Data selection}
\lb{sec:data}

he main goal of the analysis is a study of a relatively small region $\lesssim 10^\circ$ from the GC for energies $\gtrsim 1$ GeV,
we choose the \Fermi LAT Pass 8 Source class events as our main data sample.
We use 9 years of {\Fermi}-LAT data between August 4, 2008  and August 3, 2017 ({\Fermi} Mission Elapsed Time 239557418\,s--523411376\,s)
with energies between 316 MeV $ = 10^{2.5}$ MeV
and 1 TeV separated in 21 logarithmic energy bins (6 bins per decade).
We use the standard quality cuts 
To avoid contamination from cosmic ray interactions in the Earth atmosphere, 
we select events with an angle $\theta < 100^{\circ}$ with respect to the local zenith.
This zenith angle cut is sufficient for energies above 316 MeV.
We calculate the exposure and PSF using the standard {\Fermi} LAT Science Tools package version 
10-01-01 available from the {\Fermi} Science Support Center\footnote{\url{http://fermi.gsfc.nasa.gov/ssc/data/analysis/}} 
using the P8R2\_SOURCE\_V6 instrument response functions.
For spatial binning we use HEALPix\footnote{\url{http://sourceforge.net/projects/healpix/}} \citep{2005ApJ...622..759G} scheme with a pixelization of order 7  ($\approx 0\degr\!\!.46$ pixel size). 

\section{Modeling of the \Fermi bubbles at low latitudes}
\label{sec:Modeling}
One of the main problems in the analysis of the FB near the GC is the 
presence of the foreground emission components, 
such as the interactions of cosmic rays with the interstellar gas and radiation fields.
In order to test the possible effects of the foreground emission modeling,
we use several methods to estimate the contribution of the foreground emission to the data.

In particular, there is a tentative
displacement of the FB to the right of the GC, e.g., negative Galactic latitudes \citep{2016ApJS..223...26A, 2017ApJ...840...43A},
with a spectrum that is harder than the spectrum of the FB at high latitudes \citep{2017ApJ...840...43A}.
If we assume that the Galactic emission components are approximately symmetric with respect to the GC,
then we can simply mask PS and calculate the difference in gamma-ray flux to the left and to the right from the GC.
The difference should be approximately equal to the asymmetric part of the FB emission
under the assumption that the other Galactic components and unresolved PS are symmetric with respect to the GC
(Section \ref{sec:data_diff}).

In order to further test the hypothesis of the asymmetric and hard emission from the FB at low latitudes,
we use the data at energies $\lesssim \SI{1}{GeV}$ to create a template of the Galactic emission,
provided that the expected contribution of the FB at these energies is small relative to the rest of the Galactic components.
Then we fit the template derived from the low energy data together with an isotropic template at higher energies
outside of the FB area.
The FB intensity is determined by extrapolating the model inside the FB area using the full template and by subtracting the model
from the data (Section \ref{sec:le_data_model}).
As an alternative approach, instead of fitting outside of the FB area, we add independent flat rectangular templates with the size
approximately following the FB size to the model and fit over the whole sky.
The flux attributed to these rectangular templates is used as an estimate of the average flux in the FB in the corresponding areas
(Section \ref{sec:box_model}).

We also calculate the flux attributed to the bubbles using one of the diffuse emission models from \citep{2017ApJ...840...43A}
(Section \ref{sec:galprop_model}).



\subsection{Left-right difference in the data}
\label{sec:data_diff}

As a first simple check of the assymmetry at low latitudes, we compare the unprocessed \Fermi-LAT data east and west of the Galactic center. After masking PS, we average the data over a region west, i.e. longitudes $\ell \in (-10^\circ, 0^\circ)$, and east, i.e. $\ell \in (-10^\circ, 0^\circ)$, of the Galactic center for different latitudes. The regions have a width of $10^\circ$ in latitude for high latitudes, $b >|10^\circ|$, and $4^\circ$ for low latitudes, $b <|10^\circ|$. The difference of the averaged diffuse emission west $-$ east is shown in Fig. \ref{fig:data_diff} as a function of energy. At high latitudes, $b >|10^\circ|$, the emission is very symmetric. The emission for latitudes $b \in (-6^\circ, -2^\circ)$ and $b \in (-2^\circ, 2^\circ)$ shows excessive emission west of the Galactic center, which does not decay for high energies. 


\begin{figure}[h]
 \includegraphics[width=0.5\textwidth]{plots/Difference_data_for_different_latitudes.pdf}
 \caption{Difference west $-$ east in the unprocessed \Fermi-LAT data with symmetric PS mask. The region west reaches from $-10^\circ$ to $0^\circ$ longitude, the region east from $0^\circ$ to $10^\circ$ longitude.}
 \label{fig:data_diff}
\end{figure}

\subsection{Low energy data as a background model}
\label{sec:le_data_model}

\begin{figure*}[t]
	\makebox[\textwidth][c]{
    	\begin{subfigure}{0.3\textwidth}
        	\includegraphics[width=\textwidth]{plots/Mollweide_LowE_03-10GeV_flux_source_range_0.pdf}
    	\end{subfigure} 
    	\begin{subfigure}{0.3\textwidth}
        	\includegraphics[width=\textwidth]{plots/Mollweide_LowE_03-10GeV_flux_source_range_1.pdf}
    	\end{subfigure}
    	\begin{subfigure}{0.3\textwidth}
        	\includegraphics[width=\textwidth]{plots/Mollweide_LowE_03-10GeV_flux_source_range_2.pdf}
    	\end{subfigure}
    }
  	\caption{Residuals of the low-energy model for three different energy ranges. The \Fermi bubbles are clearly visible in the first two energy ranges. Point sources are masked.}
  	\label{fig:Maps_lowE}
\end{figure*}

Gamma rays produced in interactions of CR with gas and IC scattering dominate the gamma-ray emission around the GC in the energy range $E \lesssim \SI{1}{GeV}$. 
Consequently, low-energy \Fermi-LAT data is a good tracer for diffuse gamma-ray emission in the Galactic plane and can be used to create a spatial template for the Galactic foreground.
Since the angular resolution is worse for smaller energies, we smooth the data in each high-energy bin with a Gaussian kernel of $1^\circ$ to compensate for the difference in angular resolution. 

We cut the sky horizontally in latitude stripes with the width of $4^\circ$ and define our model in each stripe and energy bin separately. 
In the latitude stripe $\ell$ and energy bin $E$ our model consists of a term proportional to the low-energy photon counts $k_{\ell\alpha} \cdot \tilde N^\low_\ell(E,x)$, summed over all energies in 0.3 - $\SI{1.0}{GeV}$, and an additional term $\tilde c_{\ell}(E,x)$: 
\be
N^\model_{\ell}(E,x) = k_{\ell}(E) \cdot \tilde N^\low_{\ell}(E,x) +\tilde c_{\ell}(E,x).
\ee
The term $\tilde c_\ell(E,x)$ consists of a factor $c_\ell(E)$, which is constant in each latitude $\ell$ and energy $E$, that is weighted by the exposure $\tau(E,x)$:
\be
\tilde c_\ell(E,x) = c_\ell(E) \cdot \tau(x,E).
\ee
It takes into account the isotropic extragalactic background and partially compensates for the latitude dependent IC emission. To include the dependence of exposure on energy and positon in the sky, the low-energy data is weighted by the quotient of exposure $\tau(E,x)$ in the low- and high-energy range in each pixel $x$:
\be
\tilde N^\model_\ell(E,x) = \frac{1}{n_\low} \left(\sum_{\epsilon \in (0.3 - \SI{1.0}{GeV})} \frac{N^\low(\epsilon, x)}{\tau(\epsilon,x)}\right) \cdot \tau(E,x),
\ee
where $n_\low$ is the number of low-energy bins. 


We determine the parameters $c_{\ell}(E)$ and $k_{\ell}(E)$ by fitting the model to the \Fermi-LAT data in energy bins $E > \SI{1.0}{GeV}$
using Poisson likelihood (with Python iminuit minimizer). Since we smooth the data before the fit, the Poisson log-likelihood is an approximation in this case. To avoid an overcompensation of the \Fermi bubbles the region $-20^\circ < \ell < 20^\circ$ is excluded from the fit. 
We mask the 200 brightest 3FGL PS  with a circle of radius $\frac{\delta}{\sqrt{2}} + 1^\circ$ where $\delta = 0\degr\!\!.46$ is the characteristic size of the pixels. 
We also symmetrize the PS relative to the GC in order to avoid possible bias by masking more pixels on one side of the GC.

After we fit the model in each latitude stripe, we interpolate it inside the bubbles ROI and find the residual by subtracting it from the data.
Figure \ref{fig:Maps_lowE} shows the residual maps for three different energy ranges. 
The FB are clearly visible in the first two energy ranges, $E = 1 - \SI{10}{GeV}$ and $E = 10 - \SI{100}{GeV}$, for 
$E = \SI{100}{GeV} - \SI{1}{TeV}$ the statistics is low, but one can still see an excess near the GP.

\subsection{Rectangles model of the bubbles}
\label{sec:box_model}

Our first simple ansatz for a model of the FB consists of rectangular templates that approximately cover the area of the FB. As a model for the foreground we again use the low-energy model from Section \ref{sec:le_data_model}. As before, the fit is performed independently in each $4^\circ$ latitude stripe. To explore the east-west assymetry of the FB, two rectangular templates, one east ($-20^\circ - 0^\circ$) and one west ($0^\circ - 20^\circ$), are added to the low-energy model in each latitude stripe $\ell$ and energy bin $E$: 
\be
\begin{split}
N^\model_{\ell}(E,x) &= k_{\ell}(E) \cdot \tilde N^\low_{\ell}(E,x) +\tilde c_{\ell}(E,x)\\
&\quad + R^\east_\ell(E) + R^\west_\ell(E).
\end{split}
\ee
We determine the normalization of the rectangles $R^\east_\ell(E)$ and $R^\west_\ell(E)$ and the parameters $k_{\ell}(E)$ and $c_{\ell}(E)$ by fitting the model to the \Fermi-LAT data in energy bins $E > \SI{1.0}{GeV}$. The resulting residual, shown in Fig. \ref{fig:Maps_Rectangles} for $E = 10 - \SI{100}{GeV}$, is very similar to the low-energy model.

\begin{figure}[h]
 \includegraphics[width=0.5\textwidth]{plots/Mollweide_Boxes_residual+boxes_03-10GeV_flux_source_range_1.pdf}
 \caption{Rectangles-model residual.}
 \label{fig:Maps_Rectangles}
\end{figure}
%
%\begin{figure*}
%	\makebox[\textwidth][c]{
%    	\begin{subfigure}{0.3\textwidth}
%        	\includegraphics[width=\textwidth]{plots/Mollweide_Boxes_residual+boxes_03-10GeV_flux_source_range_0.pdf}
%    	\end{subfigure} 
%    	\begin{subfigure}{0.3\textwidth}
%        	\includegraphics[width=\textwidth]{plots/Mollweide_Boxes_residual+boxes_03-10GeV_flux_source_range_1.pdf}
%    	\end{subfigure}
%    	\begin{subfigure}{0.3\textwidth}
%        	\includegraphics[width=\textwidth]{plots/Mollweide_Boxes_residual+boxes_03-10GeV_flux_source_range_2.pdf}
%    	\end{subfigure}
%    	}
%  	\caption{Rectangles-model residuals in three different energy ranges.}
%  	\label{fig:Maps_Rectangles}
% \end{figure*}
%

\subsection{GALPROP model of the foreground and PS refitting}
\label{sec:galprop_model}


\begin{figure}[h]
 \includegraphics[width=0.5\textwidth]{plots/Mollweide_GALPROP_source_range2.pdf}
 \caption{GALPROP-model residuals in three different energy ranges.}
 \label{fig:Maps_GALPROP}
\end{figure}
%
%
%\begin{figure*}
%	\makebox[\textwidth][c]{
%    	\begin{subfigure}{0.3\textwidth}
%        	\includegraphics[width=\textwidth]{plots/Mollweide_GALPROP_source_range1.pdf}
%    	\end{subfigure} 
%    	\begin{subfigure}{0.3\textwidth}
%        	\includegraphics[width=\textwidth]{plots/Mollweide_GALPROP_source_range2.pdf}
%    	\end{subfigure}
%    	\begin{subfigure}{0.3\textwidth}
%        	\includegraphics[width=\textwidth]{plots/Mollweide_GALPROP_source_range3.pdf}
%    	\end{subfigure}
%    	}
%  	\caption{GALPROP-model residuals in three different energy ranges.}
%  	\label{fig:Maps_GALPROP}
% \end{figure*}
\section{Morphology and spectrum of the gamma-ray emission at the base of the FB}


\subsection{Latitude profiles}
\label{sec:Latitude_profiles}

In order to quantitatively estimate the difference in the intensity of emission of the FB at high latitudes and at the base as well as the asymmetry
of the gamma-ray emission at the base of the FB,
we plot the latitude profiles to the left and to the right of the GC.
In Figure \ref{fig:Profiles} we show the flux for different models integrated in the energy range $10 - \SI{100}{GeV}$ as a function of Galactic latitude. 
The regions have a width of $\ang{4}$ for $|b|<\ang{10}$ and $\ang{10}$ for $|b|>\ang{10}$ in latitude and a width of $\ang{10}$ in longitude, that is $\ang{0} - \ang{10}$ to the East of the GC an $\ang{-10} - \ang{0}$ to the West of the GC. The profiles of the three models show the residual including the FB, i.e. for the low-energy model this is the residual, for the rectangles model it is the sum of residual and rectangles template, and for the GALPROP model it is the sum of residual, FB template and GC excess template, as decsribed in section \ref{sec:Modeling}. For comparison we show the data outside of the PS mask.
%We observe an agreement between all models and a East-West assymmetry close to the GC. 
For positive longitudes,
some models overpredict the gamma-ray data which leads to oversubtration in the residuals.
For negative longitudes there is an increase of flux in the Galactic plane by at least a factor of 2 for all models relative to the gamma-ray emission
from the FB at high latitudes. 
The residual of the GALPROP model differs from the low-energy and rectangles model in the Galactic plane by a factor of $2-3$. 
This may be due to additional freedom in the GALPROP model related to many templates in the Galactic plane,
the inability of the GALPROP templates to fit the data is also manifested by the negative residuals to the East of the GC.
%Also the data shows a slight East-West assymmetry in the flux around the GC.


\begin{figure*}[h]
\includegraphics[width=0.5\textwidth]{plots/Profiles_l=1_source_range_1.pdf}
\includegraphics[width=0.5\textwidth]{plots/Profiles_l=0_source_range_1.pdf}
  	\caption{Latitude profiles of the different model residals and the data with PS mask in integrated flux in $10 - \SI{100}{GeV}$. The width of the regions in longitude is $\ang{10}$, i.e. $\ang{0} - \ang{10}$ to the East of the GC (left) and $\ang{-10} - \ang{0}$ to the West of the GC (right).}
  	\label{fig:Profiles}
\end{figure*}

\subsection{Comparison of the spectra at different latitudes}

\begin{figure*}[h]
\includegraphics[width=0.5\textwidth]{plots/SED_all_models_source_l=5_b=0.pdf}
\includegraphics[width=0.5\textwidth]{plots/SED_all_models_source_l=-5_b=0.pdf}
  	\caption{SED of the model residuals, the data with PS mask and the difference in the data with PS mask West minus East. The width of the regions in longitude is $\ang{10}$, i.e. $\ang{0}$ to $\ang{10}$ to the East of the GC (left) and $\ang{-10}$ to $\ang{0}$ to the West of the GC (right).}
  	\label{fig:SED_all}
\end{figure*}

In this section we quantify the hardening of the gamma-ray spectrum at the base of the FB. 
We first compare the SED of the different model residuals in Figure \ref{fig:SED_all}. 
The differential flux is averaged over regions to the East $(\ang{0} - \ang{10})$ and West $(\ang{-10} - \ang{0})$ of the GC, 
in a very thin stripe covering the Galactic plane $b \in (\ang{-2},\ \ang{2})$. 
For comparison we show the data with PS mask and the difference in the data West minus East, 
which is positive in every energy bin in the shown latitude band.

For negative longitudes, all models give similar results. We again observe that the differential flux of the GALPROP model is smaller than the differential flux of the other models, wich is consistent with the observation in section \ref{sec:Latitude_profiles}. We also find an assymmetry in the flux of the data with PS mask. The difference of the data West minus East is similar to the flux of the low-energy and rectangles model. The spectra at positive longitudes show large oversubtractions and an overall softer spectrum. %\Laura{Should we show here the plots for (-6,-2) and (2,6) deg latitude also?}
%\dima{I'm not sure, it will take space}

To compare the behavior of the energy spectra at high energies for different latitudes, 
we fit a log-parabola

 \be
 f(E) = N_0 \left(\frac{E}{\SI{1}{GeV}}\right)^{-\alpha - \beta \ln(E)}
 \ee
in each latitude stripe. The local ``index'' of the spectrum at energy $E$ is

 \be 
n \equiv \frac{\de \ln f}{\de \ln E} = -\alpha - 2 \beta \ln\left(\frac{E}{\SI{1}{GeV}}\right).
 \ee
In Figure \ref{fig:logpar_index} we compare this log-parabola index $n$ as a function of latitude at $E = \SI{500}{GeV}$. 
%\Laura{You called it bubble! ;)} \dima{bubble removed}
We plot $(2 - n)$, which corresponds to the SED index.
For positive longitudes the index is relatively soft ($n > 2$) for most of the latitudes, 
except high latitudes where the gamma-ray statistics is small.
For negative longitudes the index near the GC ($n \approx 2$) 
is significantly harder than the index at higher latitudes.
%\Laura{Would you like the x-axis of the log-par plot to show n-2?}
%\dima{yes, I'd suggest to put (2 - n) on the y-axis}
\begin{figure*}[h!]
\includegraphics[width=0.5\textwidth]{plots/LogParabola_n(500GeV)_l_in_(0,10).pdf}
\includegraphics[width=0.5\textwidth]{plots/LogParabola_n(500GeV)_l_in_(-10,0).pdf}
  	\caption{Index of the log-parabola (as defined in the text) at an energy $E\ = \SI{500}{GeV}$ as a function of latitude for the different model residuals. The width of the regions in longitude is $\ang{10}$, i.e. $\ang{0} - \ang{10}$ to the East of the GC (left) and $\ang{-10} - \ang{0}$ to the West of the GC (right).}
  	\label{fig:logpar_index}
\end{figure*}


\subsection{Parametric model of the gamma-ray spectrum at low latitudes}
\label{sec:param_model}

In this section we study in detail the spectrum of the FB at latitudes $|b| < 6^\circ$.
We compare a power-law model with a power-law and an exponential cutoff and find 
the 95\% confidence level for the cutoff energy.
As a baseline model we use the rectangles model of the FB, for the 95\% confidence level 
we also find the minimum among all models.

\begin{table*}
  \begin{center}
    \caption{Parametric model of the FB. We report the best fit values of the spectrum, the significance of the cutoff, 
    and statistical 95\% confidence on $E_{\rm cut}$ for the rectangles model.
    The last column is the minimum among all models of the 95\% statistical confidence value of $E_{\rm cut}$.}
    \label{tab:param}
    \begin{tabular}{|c|c|c|c|c|c|c|c|} % <-- Alignments: 1st column left, 2nd middle and 3rd right, with vertical lines in between
     	\hline
		 Lat & Lon  & norm & index & cutoff &  $-2 \Delta \log \La$ & cutoff 95\% stat & cutoff 95\% min \\ 
		       &        &  $\SI{e-6}{\frac{GeV}{cm^{2}\, s\, sr}}$ &  & $\SI{}{TeV}$ & & $\SI{}{GeV}$ & $\SI{}{GeV}$ \\ 
		\hline
  		$(\ang{2}, \ang{6})$ & $(\ang{0}, \ang{10})$ & 1.9  & -1.9 & 0.045 & 4.8 & 25 & 25 \\ 
		& $(\ang{-10}, \ang{0})$ & 2.0  & -2.2 & 5.5 & 0.0 & 180 & 180  \\ 
 		\hline
  		$(\ang{-2}, \ang{2})$ & $(\ang{0}, \ang{10})$  & 3.5  & -2.3 & 2.5 & 0.0 & 140 & 2.4  \\ 
		& $(\ang{-10}, \ang{0})$  & 6.4  & -2.1 & $\infty$ & 0.0 & 360 & 360   \\ 
 		\hline
  		$(\ang{-6}, \ang{-2})$ & $(\ang{0}, \ang{10})$  & 1.8  & -2.1 & 0.26 & 3.8 & 120 & 11  \\ 
		& $(\ang{-10}, \ang{0})$ & 2.8  & -2.2 & 440 & 0.0 & 180 & 180 \\ 
 \hline
    \end{tabular}
  \end{center}
\end{table*}




\subsection{IC model of the gamma-ray emission}
\label{sec:IC_model}

In this section we model the gamma-ray emission at the base of the FB via the IC scattering.
As a baseline case, we take the gamma-ray spectrum derived in the rectangles model of the FB in Section \ref{sec:box_model}.
The source function for the IC gamma-rays is

\be
\label{eq:IC_spectrum}
E_\g\frac{\de Q_{\rm IC}}{\de E_\g} = c\int\!\! \int \left(\frac{\de n}{\de E}\right)_{\!\!\ISRF} \sigma_\IC\ \left(\frac{\de n}{\de E}\right)_{\!\!\el} \de E_\ISRF\, \de E_\el,
\ee
where $(\de n/ \de E)_\ISRF\ [\SI{}{GeV^{-1} cm^{-3}}]$ is the number density of ISRF photons,
$(\de n / \de E)_\el\ [\SI{}{GeV^{-1} cm^{-3}}]$ is the number density of electrons, and $\sigma_\IC(E_\gamma, E_\ISRF, E_\el)$
is the differential IC scattering cross section in units of $E_\g\frac{d\sigma}{d E_\g}$ \citep{1970RvMP...42..237B}.
The ISRF number density for starlight and IR photons is taken from GALPROP v54.
For the CMB, we use the thermal spectrum with the temperature $\SI{2.73}{K}$.
The SED intensity of gamma rays is
\be
E^2 \frac{dF}{dE} = \frac{1}{4 \pi} \int E^2\frac{\de Q}{\de E} dR = 
\frac{Ec}{4\pi}\int \int \left(\frac{\de n}{\de E}\right)_{\!\!\ISRF} \sigma_\IC\ \left(\frac{\de \Sigma}{\de E}\right)_{\!\!\el} \de E_\ISRF\, \de E_\el,
\ee
where $\left(\frac{\de \Sigma}{\de E}\right)_{\!\!\el} = \int \left(\frac{\de n}{\de E}\right)_{\!\!\el} dR$ is the column density 
of CR electrons.
We will model the column density of electrons as a power law with a cutoff
\be 
\label{eq:e_spectrum}
\left(\frac{\de \Sigma}{\de E}\right)_{\!\!\el} = n_\el \left(\frac{E_\el}{\SI{1}{GeV}}\right)^{-\gamma_\el} e^{- E / E_{\cut}}.
\ee
We determine the normalization $n_\el$, spectral index $\gamma_\el$, and the cutoff  $E_{\cut}$ by fitting the IC model of the FB plus the foreground model to the 
total gamma-ray data using Poisson likelihood. 
The best-fit parameters for the rectangles model of the bubbles are reported in Figure \ref{fig:SED_with_fits}
%Figure \ref{fig:SED_with_fits} shows the residual of the rectangles model within 
in latitude stripes $b \in (\ang{2}, \ang{6})$, $b \in (-\ang{2}, \ang{2})$ and $b \in (-\ang{6}, -\ang{2})$ (as in Figure \ref{fig:SED_all}). 
%The dotted line represents the best-fit IC spectrum for an electron distribution following a simple power law.
%\Laura{Should we here compare with the spectral indices of the other models?} \dima{yes}
%The spectral index to the West of the GC of the electron spectrum is harder than the spectrum to the East of the GC. 
If the improvement in $-2 \Delta \log \La$ with and without the cutoff is less than about 1, then we show only the parameters for the power-law model without a cutoff.
For example, for negative longitudes the cutoff is not significant.

The 95\% statistical lower limit on the cutoff value for the rectangles model and the minimum among all models of the 95\% confidence values for the cutoff are presented in Table \ref{tab:IC}.
%We also report the 95\% lower limit on the cutoff value. 
For negative longitudes in the Galactic plane,
the 95\% confidence lower limit on the cutoff in the spectrum of electrons is 13 TeV,
while the minimal value of the 95\% confidence limit for all the models of the foreground emission is about 3 TeV.


\begin{comment}
For that, we determine the photon counts detected by \Fermi-LAT that correspond to the IC radiation generated by the electrons in the respective region. For a volume $V$ and a distance $R$ to the region, the detected counts per energy bin $E$ are 
\be
N_{\gamma,\IC}(E) = \left(E\frac{\de n}{\de E}\right)_{\!\!\gamma,\IC} \cdot V \frac{\tau(E_\gamma)}{4 \pi R^2} \cdot \de(\log E_\gamma),
\ee
where $ \de(\log E_\gamma)$ is the logarithmic size of the energy bin. Since the exposure $\tau(E_\gamma)$, which is averaged over the area on the sky, depends on energy, it can affect the shape of the electron spectrum. The quantities $V$ and $R$ only affect the normalization of the electron spectrum and will not be important until section \ref{sec:Interpretation}.
The model for the total detected counts is the sum of one of the foreground models and the counts generated by the electron density via IC scattering. As our baseline model for the foreground, we pick the rectangles model.
 We fit our model of the total counts to the actually observed total counts in that region (with PS mask) using Poisson likelihood and extract the parameters $n_\el$ and $\gamma_\el$ of the electron spectrum.
\end{comment}


\begin{comment}
For $b \in (\ang{2}, \ang{6})$ the spectral index varies between 2.92 and 3.15 for the three models, for $b \in (-\ang{2}, \ang{2})$ between 2.68 and 3.54 and for $b \in (-\ang{6}, -\ang{2})$ between 2.85 and 3.00. The softest spectrum in each latitude stripe is fitted to the GALPROP model. To the East of the GC the spectral indices vary between 2.97 and 5.09 in the three latitude stripes. 

In order to test the presence of a cutoff in the spectrum of electrons, we fit the gamma-ray data using a spectrum of the electrons
with an additional cutoff factor $\exp(-\alpha E_\el)$, where $\alpha = E^{-1}_{\el,\cut}$ is the inverse cutoff.
The improvement in the $-2 \log \La$ \dima{We use Poisson log likelihood, right? I think it will be less confusing to use $-2 \log \La$,
which we actually calculate, rather  than $\chi^2$. We could even replace the two $\chi^2$ columns in the tables with a single column $-2 \Delta \log \La$,
since the actual values of $-2 \log \La$ do not mean much.}
for the models with and without the cutoff is shown in Table \ref{tab:IC}
for different latitude stripes.
\end{comment}



%We want to estimate the probability for the electron spectrum to have an exponential cutoff. For that we multiply an exponential cutoff $\exp(E_\el / E_{\el,\cut})$ to the electron spectrum \eqref{eq:e_spectrum} and determine the parameters analog to the procedure described before, using the rectangles model as the baseline foreground model.\\
%For the latitude stripe covering the Galactic plane, $b \in (-\ang{2}, \ang{2})$, adding a cutoff to the powerlaw does not improve the $\chi^2$-value both at negative ($\chi^2 \approx 135$) and positive longitudes ($\chi^2 \approx 128$).

%We determine the lower bound for the cutoff energy at a $\SI{95}{\percent}$-confidence level for our baseline model, the value in parenthesis gives the lowest value for all models: For negative longitudes we find a lower bound for the cutoff energy at $\SI{13.3}{TeV}$ ($\SI{2.9}{TeV}$), for positive longitudes at $\SI{491}{GeV}$ ($\SI{16}{GeV}$).

%Slightly below the Galactic plane, $b \in (-\ang{6}, -\ang{2})$, the $\chi^2$-value does not improve by exchanging the simple powerlaw by a powerlaw with a cutoff at negative longitudes ($\chi^2 \approx 76$). At positive longitudes the $\chi^2$-value improves slightly by adding the cutoff ($\chi^2 = 98$ to $\chi^2 = 87$). For negative longitudes we find a lower bound for the cutoff energy at $\SI{6.89}{TeV}$ ($\SI{6.89}{TeV}$), for positive longitudes at $\SI{818}{GeV}$ ($\SI{0.79}{GeV}$), at a $\SI{95}{\percent}$-confidence level.

\begin{figure*}[h!]
% version for the one-column style
%\begin{comment}
\includegraphics[width=0.33\textwidth]{plots/SED_boxes_source_4cutoff.pdf}
\includegraphics[width=0.33\textwidth]{plots/SED_boxes_source_0.pdf}
\includegraphics[width=0.33\textwidth]{plots/SED_boxes_source_-4cutoff.pdf}
%\end{comment}
% version for the two-column style?
\begin{comment}
    \begin{subfigure}{0.49\textwidth}
        \includegraphics[width=\textwidth]{plots/SED_boxes_source_4.pdf}
    \end{subfigure}\\
    \begin{subfigure}{0.49\textwidth}
        \includegraphics[width=\textwidth]{plots/SED_boxes_source_0.pdf}
    \end{subfigure} \\
    \begin{subfigure}{0.49\textwidth}
        \includegraphics[width=\textwidth]{plots/SED_boxes_source_-4.pdf}
    \end{subfigure}
\end{comment}
  	\caption{SED of rectangles-model residual in the latitude stripes $(\ang{2}, \ang{6})$, $(\ang{-2}, \ang{2})$ and $(\ang{-6}, \ang{-2})$ for negative (blue) and positive longitudes (red). We determine the spectral index of a powerlaw (PL), of an electron distribution emitting gamma-rays via IC and of a proton distribution emitting gamma-rays via $\pi^0$-decay.}
  	%\Laura{Should we add the spectrum for the latitude band $(\ang{2}, \ang{6})$ also, or just say that it looks similar?}
	%\blue{Dima: yes, let's add the 2 to 6 deg spectrum.}
  	\label{fig:SED_with_fits}
\end{figure*}

%
%\begin{center}
%\begin{tabular}{ |c|c|c|c|c| } 
% \hline
% lat & lon  & $\chi^2$(no cutoff) &  $\chi^2$(cutoff) & Lower bound $E_\cut$ \\ 
% \hline
%  2 -- 6 & east & $\chi^2$(no cutoff) &  $\chi^2$(cutoff) & Lower bound $E_\cut$\\ 
%2 -- 6 & west & $\chi^2$(no cutoff) &  $\chi^2$(cutoff) & Lower bound $E_\cut$ \\ 
% \hline
%   -2 -- 2 & east & $\chi^2$(no cutoff) &  $\chi^2$(cutoff) & Lower bound $E_\cut$\\ 
%-2 -- 2 & west & $\chi^2$(no cutoff) &  $\chi^2$(cutoff) & Lower bound $E_\cut$\\ 
% \hline
%  -6 -- -2 & east & $\chi^2$(no cutoff) &  $\chi^2$(cutoff) & Lower bound $E_\cut$\\ 
%-6 -- -2 & west & $\chi^2$(no cutoff) &  $\chi^2$(cutoff) & Lower bound $E_\cut$\\ 
% \hline
%\end{tabular}
%\end{center}

\begin{table*}
  \begin{center}
    \caption{$\chi^2$-values for IC-spectrum fit of a distribution of electrons following a simple powerlaw and a powerlaw with cutoff, respectively in the latitude bands discussed in the text. 
The lower bounds for $E_\cut$ at $\SI{95}{\percent}$ confidence level for our baseline model and the minimum among all
models are shown in the last two columns respectively.
\dima{I've added an extra column for the minimum of cutoff among all models, if you think it's OK, please, change the pi0 table as well.
I'd also suggest to keep only the integer values for the $\chi^2$, i.e.,  305.9 --$>$ 306, and to have at most two significant digits
in the cutoff values.}}
    \label{tab:IC}
    \begin{tabular}{|c|c|c|c|c|} % <-- Alignments: 1st column left, 2nd middle and 3rd right, with vertical lines in between
     	\hline
		 Lat & Lon  & $-2 \Delta \log \La$ & \multicolumn{2}{c|}{Lower bound on $E_\cut$ (TeV) } \\ 
		       &        &                                  &  \multicolumn{1}{c}{Rectangles model} & All models \\ 
		\hline
  		$(\ang{2}, \ang{6})$ & $(\ang{0}, \ang{10})$ & 2.6  & 0.05 & 0.05 \\ 
		& $(\ang{-10}, \ang{0})$ & 0.0  & 4.0  & 4.0 \\ 
 		\hline
  		$(\ang{-2}, \ang{2})$ & $(\ang{0}, \ang{10})$ & 0.0 & 1.5 & 0.01 \\ 
		& $(\ang{-10}, \ang{0})$ & 0.0 & 13  & 2.9  \\ 
 		\hline
  		$(\ang{-6}, \ang{-2})$ & $(\ang{0}, \ang{10})$ & 1.1 & 0.83 & 0.03 \\ 
		& $(\ang{-10}, \ang{0})$& 0.0 & 6.3 & 6.3\\ 
 \hline
    \end{tabular}
  \end{center}
\end{table*}




%\begin{figure}
%	\includegraphics[width=0.5\textwidth]{plots/SED_boxes_source_4.pdf}
%\end{figure}
%\begin{figure}
%	\includegraphics[width=0.5 \textwidth]{plots/SED_boxes_source_0.pdf}
%\end{figure}
%\begin{figure}
%	\includegraphics[width=0.5\textwidth]{plots/SED_boxes_source_-4.pdf}
%	\caption{SED of rectangles-model residual in the latitude stripes $(\ang{-2}, \ang{2})$ (left) and $(\ang{-6}, \ang{-2})$ (right) for negative (blue) and positive longitudes (red). We determine the spectral index of a powerlaw (PL), of an electron distribution emitting gamma-rays via IC and of a proton distribution emitting gamma-rays via $\pi^0$-decay.}
%\end{figure}


\subsection{Hadronic model of gamma-ray emission}
\label{sec:Pion_model}

In the hadronic model, the gamma rays are produced as a result of collisions of hadronic CR with the interstellar gas.
The source function for the gamma rays is 
\be
\left(E\frac{\de Q}{\de E}\right)_{\!\!\gamma, \pi^0}\! = \int n_\Hy\ \sigma_\pr v_\pr \left(\frac{\de n}{\de T}\right)_{\!\!\pr} \de T_\pr,
\label{eq:had_spectrum}
\ee
where the integral goes over the kinetic energies of the protons $T_\pr = \sqrt{(qc)^2 + (mc^2)^2} - mc^2$,
$n_\Hy$ is the density of gas, and $\sigma_\pr (E_\gamma, T_\pr)$ is 
the differential cross section in units of $E_\g\frac{d\sigma}{d E_\g}$
to produce gamma rays in proton-proton collisions \citep{2006ApJ...647..692K, 2008ApJ...674..278K}.
We will use $n_\Hy = \SI{1}{cm^{-3}}$ as a characteristic density,
which is consistent with the gas surface density of $\sim 10 M_\odot {\rm pc}^{-2}$ \citep{2017ApJ...834...57M}
averaged over $\approx 200$ pc above and below the GC.
We will model the proton spectrum as a power law of the momentum $\frac{d n}{d qc} = n_\pr q^{-\g_p}$ 
(note, that $ v \frac{\de n}{\de T} = c \frac{d n}{d qc}$).

We add the hadronic model of the gamma-ray emission at the base of the FB to the foreground emission 
and determine the normalization $n_\pr$ and index $\gamma_\pr$ of the CRp spectrum
by fitting the total model to the \Fermi-LAT data using Poisson log likelihood.
The dash-dotted line in Figure \ref{fig:SED_with_fits} represents the best-fit hadronic spectrum (labeled as $\pi^0$). 
The index of the proton spectrum is relatively hard $\g_\pr \lesssim 2.3$, especially to the West of the GC.
\begin{comment}
For that, we again fit the sum of the photon counts generated by hadronic processes and the baseline model for the foreground to the total photon counts detected by the \Fermi-LAT (with PS mask) using Poisson likelihood.  
For $b \in (\ang{2}, \ang{6})$ the spectral index to the West of the GC varies between 2.26 and 2.42, for $b \in (-\ang{2}, \ang{2})$ between 2.14 and 2.64 and for $(-\ang{6}, -\ang{2})$ between 2.21 and 2.32. To the East of the GC the indices vary between 2.33 and 3.55. 
\end{comment}

We calculate the significance of a cutoff in the CRp spectrum by adding an exponential cutoff factor and refitting the model to the gamma-ray data.
The improvement in the model and the 95\% confidence lower bound on the cutoff values are presented in Table \ref{tab:pi0}.
Within $\pm 2^\circ$ from the Galactic plane West of the GC, the 95\% confidence level for the lower bound on the cutoff among all the models
of the foreground emission that we have considered is about 20 TeV.


\begin{comment}
We again estimate the probability for the proton spectrum to have an exponential cutoff: For the latitude stripe covering the Galactic plane, $b \in (-\ang{2}, \ang{2})$, adding a cutoff does neither improve the $\chi^2$-valueat negative ($\chi^2 \approx 62$) nor positive longitudes ($\chi^2 \approx 116$). At a $95\%$-confidence level, the lower bound for the cutoff energy for the baseline model (and all models) is $\SI{28.6}{TeV}$ ($\SI{22.6}{TeV}$) for negative longitudes and $\SI{1.8}{TeV}$ ($\SI{11.5}{GeV}$) for positive longitudes.\\
In the latitude band $(-\ang{6}, -\ang{2})$, the $\chi^2$-value does increase both for negative ($\chi^2 = 157$ to $\chi^2 = 123$) and positive longitudes ($\chi^2 = 245$ to $\chi^2 = 156$) by adding an exponential cutoff. The lower bound on the cutoff energy is $\SI{23.6}{TeV}$ ($\SI{0.99}{TeV}$) for negative and $\SI{1.57}{TeV}$ ($\SI{40}{GeV}$) for positive longitudes.
\end{comment}


\begin{table*}
  \begin{center}
    \caption{$\chi^2$-values for hadronic-spectrum fit of a distribution of protons following a simple powerlaw and a powerlaw with a cutoff, respectively in the latitude bands discussed in the text. The lower bound for $E_\cut$ at a $\SI{95}{\percent}$-confidence level for our baseline model is shown in the last column; the value in parenthesis gives the lowest value for all models.}
    \label{tab:pi0}
    \begin{tabular}{|c|c|c|c|c|} % <-- Alignments: 1st column left, 2nd middle and 3rd right, with vertical lines in between
     	\hline
		 lat & lon  & $-2 \Delta \log \La$ & \multicolumn{2}{c|}{Lower bound on $E_\cut$ (TeV) } \\
		      &        &                                  &       \multicolumn{1}{c}{Rectangles model} & All models \\ 
		\hline
  		$(\ang{2}, \ang{6})$ & $(\ang{0}, \ang{10})$ & 4.4 & 0.16  & 0.16 \\ 
		& $(\ang{-10}, \ang{0})$ &  0.0 & 1.6 & 1.6 \\ 
 		\hline
  		$(\ang{-2}, \ang{2})$ & $(\ang{0}, \ang{10})$ & 0.0 & 1.3 & 0.023 \\ 
		& $(\ang{-10}, \ang{0})$ & 0.0 & 30 & 6.3 \\ 
 		\hline
  		$(\ang{-6}, \ang{-2})$ & $(\ang{0}, \ang{10})$ & 2.7 & 1.6 & 0.05 \\ 
		& $(\ang{-10}, \ang{0})$ & 0.0 & 2.1 & 2.1 \\ 
 \hline
    \end{tabular}
  \end{center}
\end{table*}


\subsection{Summary of the spectral analysis}
%\dima{we can put a summary plot with the baseline model and the band of all spectra in a small subsection here}

\begin{figure}[h]
\centering
 \includegraphics[width=0.48\textwidth]{plots/Summary_SED_b=0_l=-5.pdf}
 \includegraphics[width=0.48\textwidth]{plots/Summary_SED_b=0_l=5.pdf}
 \caption{SED in the Galactic plane. The shaded areas give the systematic uncertainties estimated from the variation of the other models (using 4 different low-energy ranges).}
 \label{fig:spec_summary}
\end{figure}


\begin{table*}
  \begin{center}
    \caption{Summary of the min and max models for the parametric, 
    IC and hadronic models of the FB for $|b| < 2^\circ$ and $-10^\circ < \ell < 0^\circ$. 
    For the parametric model we report the energy spectrum of the gamma rays,
    for the IC model we report the surface density of the electrons spectrum as a function of energy,
    while for the hadronic model -- the surface density of the protons spectrum as a function of momentum.
    The normalizations are provided at 1 GeV.}
    \label{tab:summary}
    \begin{tabular}{| l |c|c|c|c|c|c|c|} % <-- Alignments: 1st column left, 2nd middle and 3rd right, with vertical lines in between
     	\hline
		 {\hspace{2cm}Model} & Type  & norm & index & cutoff & cutoff 95\% stat \\ 
		       &        &   &  & {\rm TeV} & {\rm GeV}\\ 
		\hline
  		\multirow{2}{*}{Parametric, $\frac{dN_\g}{dE} = \left[{\rm \frac{1}{GeV\, cm^{2}\, s}}\right]$} & max & $\SI{7.3e-6}{}$  & -2.0 &  2.1 & 0.9 \\ 
		& min & $\SI{1.9e-6}{}$ & -2.2 &  0.9 & 0.3  \\ 
 		\hline
  		\multirow{2}{*}{IC, $\frac{d\Sigma_e}{dE} = \left[{\rm \frac{1}{GeV\,cm^2}}\right]$} & max & 1.  & -2 &  5 & 1 \\ 
		& min & 1.  & -2 &  5 & 1  \\ 
 		\hline
  		\multirow{2}{*}{Hadronic, $\frac{d\Sigma_p}{dqc} = \left[{\rm \frac{1}{GeV\,cm^2}}\right]$} & max & 1.  & -2 &  5 & 1 \\ 
		& min & 1.  & -2 &  5 & 1  \\ 
 \hline
    \end{tabular}
  \end{center}
\end{table*}

\begin{figure}[h]
\centering
 \includegraphics[width=0.48\textwidth]{plots/low_lat_FB_CTA.pdf}
 \includegraphics[width=0.48\textwidth]{plots/low_lat_FB_KM3.pdf}
 \caption{Sensitivities of CTA and KM3NeT to the hadronic model of the gamma-ray emission towards the base of the FB.
 For the flux sensitivities we take the estimates of \cite{2018APh...100...69A} for a $2^\circ$ radius source and divide by the
 area of the $2^\circ$ circle to transform to intensity units. The min and max models correspond to the min and max models in
 the hadronic scenario in Table \ref{tab:summary}. The KM3NeT sensitivity takes into account muon neutrinos only.
 In estimating the sensitivity, we take into account that the GC is below the horizon for about 2/3 of the time
 at the KM3NeT location in the Mediterranean sea.
 }
 \label{fig:sensitivities}
\end{figure}



\section{Conclusions}
We want to estimate the properties that particles need to have to emit the gamma radiation observed in the region $b \in (\ang{-2},\ang{2}),\ \ell \in (\ang{-10},\ang{0})$. Averaged over the ROI, the best-fit spectrum of the electrons shows no cutoff and we conclude, that electrons of at least $E_0 = \SI{1}{TeV}$ need to be able to travel inside the whole volume of the ROI.  We start with a diffusion equation taking into account diffusion and energy loss $b_\IC(E)$ via IC. From the solution we read off the diffusion distance for electrons with energy $E_0 = \SI{1}{TeV}$:
\be
\langle x \rangle^2 = 2 \int_{E_0}^\infty \frac{D(E)}{b_\IC(E)}\de E = \SI{1300}{pc}.
\ee
We assumed a spatially constant diffusion coefficient $D(E) = D_0\left(\frac{E}{\SI{1}{GeV}}\right)^\delta$ with values of the local diffusion coefficient $D_0 = \SI{3e28}{cm^2/s} = \SI{100}{pc^2/kyr}$ and $\delta = 0.4$. The energy loss $b_\IC(E)$ is \dots.\\
Assuming a distance of $\SI{8}{kpc}$ to the ROI, we find a height of $\SI{0.56}{kpc}$ in latitude and a length of $\SI{0.98}{kpc}$ in longitude. Since the diffusion distance of the electrons exceeds the spatial size of the ROI, the electrons cannot be confined. Therefore, we find that a transient process is favoured. The energy losses of protons exceed the energy losses of electrons by far, therefore the same applies for protons.\\
Wit the local diffusion coefficient we find an escape time for both electrons and protons of 
\be
T = \frac{\Delta x^2}{2 D(E)} = \SI{70}{kyr}.
\ee

The total energy density in electrons with energy above $E_0 = \SI{1}{GeV}$, which needs to be generated by this transient process, is given by the integral of the electron spectrum that was found in Section \ref{sec:IC_model}:

\be
\frac{\de E_\tot}{\de V} = \int_{E_0}^{\infty} \left(E \frac{\de N}{\de E}\right)_{\!\!\el} \de E = \SI{4.9}{meV/cm^3}.
\ee
Assuming that the depth along the optical axis coincides with the length in longitude, the volume of the ROI is $V = \SI{0.54}{kpc^3} = \SI{1.58e64}{cm^3}$. The total energy content of the ROI in electrons above $\SI{1}{GeV}$ is $E_\tot = \SI{8e61}{eV} = \SI{1e50}{erg}$, which corresponds to the CR energy output of $10$ SNe.\\
Using the result from Section \ref{sec:Pion_model} we find an energy density in protons of $\de E_\tot / \de V = \SI{84}{meV/cm^3}$ and a total energy content of $E_\tot = \SI{1e63}{eV} = \SI{2e51}{erg}$. However, the gas density in the inner Galaxy is probably higher than $n_\Hy = \SI{1}{/cm^3}$ as assumed in Section \ref{sec:Pion_model}, resulting in an energy content in protons of the same order of magnitude as the energy density in electrons. \\
\\


\newpage
\bibliography{gp_bubbles_papers}  

\begin{appendix}
\section{Appendix}

If we need one.
\end{appendix}

\end{document}

